%%%%%%%%%%%%%%%%%%%%%%%%%%%%%%%%%%%%%%%%%
% Medium Length Graduate Curriculum Vitae
% LaTeX Template
% Version 1.1 (9/12/12)
%
% This template has been downloaded from:
% http://www.LaTeXTemplates.com
%
% Original author:
% Rensselaer Polytechnic Institute (http://www.rpi.edu/dept/arc/training/latex/resumes/)
%
% Important note:
% This template requires the res.cls file to be in the same directory as the
% .tex file. The res.cls file provides the resume style used for structuring the
% document.
%
%%%%%%%%%%%%%%%%%%%%%%%%%%%%%%%%%%%%%%%%%

%----------------------------------------------------------------------------------------
%	PACKAGES AND OTHER DOCUMENT CONFIGURATIONS
%----------------------------------------------------------------------------------------

\documentclass[margin, 10pt]{res} % Use the res.cls style, the font size can be changed to 11pt or 12pt here

\usepackage{helvet} % Default font is the helvetica postscript font
%\usepackage{newcent} % To change the default font to the new century schoolbook postscript font uncomment this line and comment the one above

\setlength{\textwidth}{5.1in} % Text width of the document

\begin{document}

%----------------------------------------------------------------------------------------
%	NAME AND ADDRESS SECTION
%----------------------------------------------------------------------------------------

\moveleft.5\hoffset\centerline{\large\bf Accelerating machine learning through GPU and specialized processor unit.
} % Your name at the top
 
\moveleft\hoffset\vbox{\hrule width\resumewidth height 1pt}\smallskip % Horizontal line after name; adjust line thickness by changing the '1pt'

%----------------------------------------------------------------------------------------

\begin{resume}
%----------------------------------------------------------------------------------------
%	EDUCATION SECTION
%----------------------------------------------------------------------------------------
\section{PAPER}

Bingsheng He, Wenbin Fang, Qiong Luo, Naga K. Govindaraju, and Tuyong Wang. “Mars: A MapReduce Framework on Graphics Processors”. in PACT, 2008.

M. C. Díaz, F. A. González, and R. Ramos-Pollan. “Accelerating common machine learning algorithms through GPGPU symbolic computing”. In IEEE Computing Colombian Conference,  Bogota, Colombia, 2015

Chuntao Hong, Dehao Chen, Wenguang Chen, Weimin Zheng, and Haibo Lin. “MapCG: Writing Parallel Program Portable between CPU and GPU”. In PACT, 2010.

Rajat Raina, Anand Madhavan and Andrew Y. Ng. “Large-scale deep unsupervised learning using graphics processors”. In Proceeding ICML '09 Proceedings of the 26th Annual International Conference on Machine Learning, 2009.

Bryan Catanzaro, Narayanan Sundaram and Kurt Keutzer. “Fast support vector machine training and classification on graphics processors”. In Proceeding
ICML '08 Proceedings of the 25th international conference on Machine learning, 2008.

Z. Chen, J. Wang, H. He and X. Huang. “A fast deep learning system using GPU”. In Circuits and Systems (ISCAS), IEEE International Symposium on, 2014.

J. Lu, S. Young, I. Arel and J. Holleman.”A 1 TOPS/W Analog Deep Machine-Learning Engine With Floating-Gate Storage in 0.13 µm CMOS”. In IEEE Journal of Solid-State Circuits, 2015.

Peilong Li, Yan Luo, Ning Zhang and Yu Cao. “HeteroSpark: A heterogeneous CPU/GPU Spark platform for machine learning algorithms”. In Networking, Architecture and Storage (NAS), , IEEE International Conference, 2015.

Javier López-Fandiño, Pablo Quesada-Barriuso, Dora B. Heras, and Francisco Argüello. ”Efficient ELM-Based Techniques for the Classification of Hyperspectral Remote Sensing Images on Commodity GPUs”. In IEEE Journal of Applied Earth Observations and Remote Sensing, 2015.

Nguyen Thi Thanh Van and Tran Ngoc Thinh. "Accelerating anomaly-based IDS using Neural Network on GPU". In International Conference on Advanced Computing and Applications. 

\end{resume}
\end{document}